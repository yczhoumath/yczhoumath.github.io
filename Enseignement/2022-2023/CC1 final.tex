\documentclass[a4paper,10pt]{article}
\usepackage[french]{babel}
\usepackage[T1]{fontenc}
\usepackage[utf8]{inputenc}
\usepackage[margin=1in]{geometry}
\linespread{1.2}
\pagestyle{empty}
\usepackage{enumitem}

\usepackage{amsmath,amssymb,amsfonts}
\newcommand{\N}{\mathbb{N}}
\newcommand{\Z}{\mathbb{Z}}
\newcommand{\R}{\mathbb{R}}
\newcommand{\C}{\mathbb{C}}
\newcommand{\X}{\mathcal{X}}
\newcommand{\scalarprod}[2]{\langle#1,#2\rangle}
\newcommand{\norm}[1]{\|#1\|}
\newcommand{\abs}[1]{\left|#1\right|}

\begin{document}

\noindent LU2MA260 - Contrôle continu n\textdegree1
\hfill 8 novembre 2022
\medskip\hrule
\vspace{.3in}

\emph{Durée: 1 heure. Aucun document autorisé.
Ce contrôle sera noté sur \textbf{10 points}.}
% \emph{Chaque exercice sera noté sur 2 points. Si la feuille n'est pas blanche, ça fait au moins 0.5.
% Si vous n'avez pas fini, ça fait au plus 1.5. De plus, une erreur calculatoire qui ne change pas la réponse finale sera pas tenue en compte dans les notes.}

\vspace{.1in}

\noindent
\textbf{1. (2 pts)} Questions de cours:

\begin{enumerate}[label=\alph*)]
    % \item Rappeler le critère de d'Alembert pour les séries numériques.
    \item Démontrer que la série $\sum_{n\geq1}\frac{1}{n^\alpha}$ converge si $\alpha>1$.
    \item Énoncer le théorème sur la dérivabilité de la convergence uniforme d'une suite de fonctions.
\end{enumerate}

% \vspace{.1in}


% \noindent
% 3. Vrai ou faux (jusitifier s'il est vrai, en donner un contre-exemple s'il est faux):
% \begin{enumerate}[label=\alph*)]
%     \item Soit $(a_n)_{n\geq0}$ une suite dans $\R$ telle que la série $\sum_n u_n$ n'est pas convergente. Alors la somme partiel $S_n=\sum_{k\geq0}^na_k$ tend vers $+\infty$ lorsque $n\to+\infty$.
%     \item Soit $f:[a,+\infty[ \to \R^+$ une fonction continue positive telle qu'il existe $L\in\R$ (forcément positif) tel que pour tout $x\geq a$, $\int_a^xf(t)dt\leq L$. Alors l'intégrale généralisée $\int_a^{+\infty}f(t)dt$ est convergente.
% \end{enumerate}

\vspace{.1in}


\noindent
\textbf{2. (4 pts)}
Déterminer la nature des séries numérique suivantes:
\begin{enumerate}[label=\alph*)]
\begin{minipage}{0.4\linewidth}
    \item $\sum_{n\geq1}(\sqrt[n]{n}-1)$,
    \item $\sum_{n\geq1}(\sqrt[n]{n}-1)^n$,
    % \item $\sum_{n\geq1}ne^{-\sqrt{n}}$,
\end{minipage}
\begin{minipage}{0.4\linewidth}
    % \item $\sum_{n\geq1}\frac{x^n}{n^2}$, pour $x\in\R$,
    % \item $\sum_{n\geq0}\frac{3^n}{2^n+\alpha^n}$, où $\alpha\in\R_+$,
    \item $\sum_{n\geq1}(-1)^{n-1}\frac{\lambda^n}{n}$, pour $\lambda\in\R$,
    \item $\sum_{n\geq1}\frac{(3n)!}{30^n(n!)^3}$.
    % \item $\sum_{n\geq1}\frac{\sqrt{n+1}-\sqrt{n}}{n}$.
\end{minipage}
\end{enumerate}

\vspace{.1in}

\noindent
\textbf{3. (2 pts, Noyau de Fejér)} Considérons la suite de fonctions $(K_n)_{n\in\N}$ où on définit
$$K_n(x)=\frac{1}{n}\frac{\sin^2(nx/2)}{\sin^2(x/2)},\quad\forall x\in]0,\pi].$$
Déterminer la limite simple de $(K_n)_{n\in\N}$ sur $]0,\pi]$. Déterminer si la convergence est uniforme sur
\begin{enumerate}[label=\alph*)]
    \item $I=]0,\pi[$;
    \item $I=[\delta,\pi]$, où $0<\delta<\pi$.
\end{enumerate}
(Indication: majorer $\sin^2(nx/2)$ et $\frac{1}{\sin^2(x/2)}$ séparément.)

% \vspace{.1in}

% \noindent
% \textbf{3. (2 pts)} Considérons la suite de fonctions $(f_n)_{n\in\N}$ où on définit pour tout $x\in[0,+\infty[$
% $$f_n(x)=(1+\frac{x}{n})^n.$$
% Déterminer la nature de convergence (simple, uniforme) de la suite $(K_N)_{N\in\N}$ sur $I$ avec
% \begin{enumerate}[label=\alph*)]
%     \item $I=[0,+\infty[$;
%     \item $I=[0,M]$, où $M\in\R_{>0}$.
% \end{enumerate}

\vspace{.1in}

\noindent
\textbf{4. (5 pts)}
Soit $(f_n:[0,1]\to\C)_{n\in\N}$ une suite de fonctions continues. On suppose que
\begin{enumerate}[label=\roman*)]
    \item la suite $(f_n)_n$ converge simplement vers une fonction intégrable $f:[0,1]\to\C$;
    \item la convergence est uniforme sur tout compact inclus dans $]0,1]$;
    \item \textbf{(borne uniforme) il existe $M\in\R_+$ tel que pour tout $n$, on a $\norm{f_n}_\infty\leq M$.}
\end{enumerate}
Sous ces conditions,
\begin{enumerate}[label=\alph*)]
    \item Chacune des assertions suivantes est-elle vraie ou fausse ? Démontrer si elle est vraie; en fournir un contre-exemple avec justification si elle est fausse.
    \begin{itemize}
        % \item[a.1)] $f$ est continue sur $]0,1]$.
        \item[a.1)] la limite $f$ est continue sur $[0,1]$.
        \item[a.2)] la suite $(f_n)_n$ converge uniformément sur $]0,1]$. 
        \item[a.3)] Si $f$ est continue sur $[0,1]$, alors la convergence est uniforme sur $[0,1]$.
    \end{itemize}
    \item  Soit $\delta\in]0,1[$. Montrer que
    $$\lim_n\int_\delta^1f_n(t)dt=\int_\delta^1f(t)dt$$
    et que
    $$\abs{\int_0^\delta f_n(t)dt-\int_0^\delta f(t)dt}\leq 2\delta M.$$
    \item En déduire que
    $$\lim_n\int_0^1f_n(t)dt=\int_0^1f(t)dt.$$
\end{enumerate}

% \vspace{.1in}

% \noindent
% \textbf{4. (3 pts, Fonction polylogarithme $Li_2(z)$)} On considère la série de fonctions
% $$f(x):=\sum_{n\geq1}\frac{x^n}{n^2}$$
% pour $x\in\R$.
% \begin{enumerate}[label=\alph*)]
%     \item Montrer que $f(x)$ converge normalement sur $[-1,1]$.
%     \item Montrer que $f(x)$ est dérivable sur $]-1,1[$ et calculer $xf'(x)$.
%     \item Montrer que $f(x)$ est deux fois dérivable sur $]-1,1[$ et qu'elle vérifie l'équation différentielle ordinaire
%     $$x(1-x)f''(x)+(1-x)f'(x)=x.$$
% \end{enumerate}

\vspace{.1in}

\noindent
\textbf{5. (2 pts, Bonus)}
Soit $\sum_n a_n$ une série divergente avec $a_n>0$. Notons $S_n=\sum_{k\leq n}a_k$. Montrer que $\sum_n \frac{a_n}{S_n}$ diverge. (Indication: le critère de Cauchy.)









\end{document}


