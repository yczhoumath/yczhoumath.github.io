\documentclass[a4paper,10pt]{article}
\usepackage[french]{babel}
\usepackage[T1]{fontenc}
\usepackage[utf8]{inputenc}
\usepackage[margin=1in]{geometry}
\linespread{1.2}
\pagestyle{empty}
\usepackage{enumitem}

\usepackage{amsmath,amssymb,amsfonts}
\newcommand{\N}{\mathbb{N}}
\newcommand{\Z}{\mathbb{Z}}
\newcommand{\R}{\mathbb{R}}
\newcommand{\C}{\mathbb{C}}
\newcommand{\X}{\mathcal{X}}
\newcommand{\scalarprod}[2]{\langle#1,#2\rangle}
\newcommand{\norm}[1]{\|#1\|}
\newcommand{\abs}[1]{\left|#1\right|}
\let\Re\relax
\DeclareMathOperator{\Re}{Re} % real part

\usepackage{xcolor}
\newcommand{\correction}[1]{{\color{red}#1}}
\newcommand{\comment}[1]{{\color{blue}#1}}

\begin{document}

\noindent LU2MA260 - Contrôle continu n\textdegree2
\hfill 13 décembre 2022
\medskip\hrule
\vspace{.3in}

\emph{Durée: 1 heure. Aucun document autorisé.
Ce contrôle sera noté sur \textbf{10 points}. Justifier TOUT.}


\vspace{.1in}
\noindent
\textbf{1.}
Déterminer le rayon de convergence des séries entières suivantes:
\begin{enumerate}[label=\alph*)]
\begin{minipage}{0.4\linewidth}
    \item $\sum_n 2^nz^n$,
    \item $\sum_n 2^nz^{2n}$,
    \item $\sum_n 2^nz^{n^2}$,
\end{minipage}
\begin{minipage}{0.4\linewidth}
    \item $\sum_n n^{\ln n}z^n$
    \item $\sum_n \frac{(2n)!}{(n!)^2}z^n$,
    \item $\sum_n n\sin(\frac{n\pi}{3})z^n$.
\end{minipage}
\end{enumerate}

\correction{
\begin{enumerate}[label=\alph*)]
    \item Appliquer le critère de d'Alembert : $\rho=\frac{1}{2}$.
    \item Appliquer d'Alembert généralisé ou utiliser TD 4 Exercice 3 (1) : $\rho=\frac{1}{\sqrt{2}}$.\\
    \comment{On ne peut pas appliquer le critère de d'Alembert (la version non généralisé) directement car LES COEFFICIENTS DE $z^{2n+1}$ SONT NULS.\\
    D'ailleurs, si vous faites le changement de variables $n\mapsto m/2$, le problème c'est que la série obtenue devient $\sum_{m~\text{pair}}2^{m/2}z^m$; or, la série $\sum_{m~\text{pair}}\abs{2^{m/2}z^m}$ est CONTROLEE PAR, MAIS PAS EQUIVALENTE A, $\sum_{m\in\N}\abs{2^{m/2}z^m}$; ainsi le critère de Cauchy-Hadamard ou de d'Alembert donnent seulement que la série en question converge si $\abs{z}<1/\sqrt{2}$; il faudrait encore justifier la divergence ailleurs.}
    \item Par définition du rayon de convergence du polycopié ou TD 4 Exercice 5 : la suite $(2^nr^{n^2})_n$ est bornée si $r<1$ et non bornée si $r>1$, d'où $\rho=1$.\\
    \comment{On ne peut non plus appliquer le critère de d'Alembert. D'ailleurs, si vous voulez appliquer le critère de d'Alembert << ponctuellement >>, alors il faudrais regarder le comportement de $\abs{2^{n+1}z^{(n+1)^2}/2^nz^{n^2}}=\abs{2z^{2n+1}}$ À LA LIMITE $n\to+\infty$. Voici l'argument:}
    \correction{Si $z=0$, la série converge; sinon on applique le critère de d'Alembert ponctuellement pour la série $\sum_n2^nz^{n^2}$ : si $0<\abs{z}<1$, $\lim_n\abs{2z^{2n+1}}=0<1$, d'où la convergence; si $\abs{z}>1$, $\lim_n\abs{2z^{2n+1}}=+\infty>1$, d'où la divergence.}
    \comment{Attention: si $\abs{z}=1$, $\lim_n\abs{2z^{2n+1}}=2>1$, donc la série diverge également.}
    \item Appliquer Cauchy-Hadamard :  $n^{(\ln n)/n}=e^{(\ln n)^2/n}\to e^0=1$, d'où $\rho=1$.
    \item Appliquer d'Alembert pour trouver $\rho=1/4$.\\
    \comment{$(2n)!$ devient $(2(n+1))!=(2n+2)(2n+1)\cdot(2n)!$ lorsqu'on remplace $n$ par $n+1$.}
    \item Par définition du rayon de convergence du polycopié ou TD 4 Exercice 5: d'une part, la suite $(n\sin(n\pi/3)z^n)_n$ est non bornée si $\abs{z}\geq1$ car la sous-suite $((6n+1)\sin(\pi/3)z^{6n+1})_n$ l'est, donc $\rho\leq 1$; d'autre part, $\abs{n\sin(n\pi/3)z^n}\leq n\abs{z}^n$, qui forme une suite bornée si $\abs{z}<1$, donc $\rho\geq1$. Ainsi $\rho=1$.\\
    \comment{On ne peut pas appliquer d'Alembert car les coefficients S'ANNULENT si $n$ est un multiple de $6$. La périodicité de $\sin(n\pi/3)$ vous aide seulement à trouver une sous-suite non bornée si $\abs{z}>1$. Notez que $\sin(6n\cdot\pi/3)$ s'annule toujours pour $n\in\N$.}
\end{enumerate}
}

\vspace{.1in}
\noindent
\textbf{2.} Exemples.
\begin{enumerate}[label=\alph*)]
    \item Donner une série entière dont le domaine de convergence est exactement $D(0,1)$.\\
    \correction{La série géométrique $\sum_n z^n$ convient.}\\
    \comment{La série $\sum_n z^n/n$ a pour DOMAINE DE CONVERGENCE $\overline{D}(0,1)\backslash\{1\}$ : surtout faites attention quand $z=e^{i\theta}\neq 1$, utilisez le critère d'Abel, cf. TD 1.}
    \item Donner une série entière dont le domaine de convergence est exactement $\overline{D}(0,1)$.\\
    \correction{La série $\sum_nz^n/n^2$ convient puisqu'elle converge normalement sur $\overline{D}(0,1)$, et diverge si $\abs{z}>1$.}\\
    \comment{Mais la série $\sum_nz^n/n!$ ne convient pas car elle converge trop vite de sorte que son domaine de convergence est $\C$, qui est beaucoup plus gros que demandé. Non plus pour la série $\sum_nz^n/n$ ou $\sum_n(-1)^nz^n/n=\sum_n(-z)^n/n$, voir la remarque précédente.}
\end{enumerate}

\vspace{.1in}
\noindent
\textbf{3.}
En appliquant le théorème d'Abel radial,
montrer que $\sum_{n=1}^{+\infty} \frac{(-1)^n}{n}=-\ln(2)$.

\correction{On a 
$$-\ln(1+x)=\sum_n (-1)^nx^n/n,\quad \forall x\in]-1,1[,$$
que vous l'écriviez directement ou la montriez par intégration.}
\comment{Mais attention c'est un intervalle OUVERT a priori et que $x=1$ N'EST PAS DANS CET INTERVALLE ! Parce qu'en général, les résultats de la dérivation, d'intégration ne marchent que dans le disque OUVERT $D(0,\rho)$. Pour justifier que l'on peut évaluer l'identité ci-dessus en $x=1$, on a besoin du théorème d'Abel radial.}
\correction{La série à gauche ci-dessus converge si $x=1$ (série alternée), donc Abel radial nous dit que la série  $\sum_n(-1)^nx^n/n$ converge uniformément sur $[0,1]$ avec $x=1$ contenu dedans, donc la somme est CONTINUE sur $[0,1]$, donc
$$\sum_n(-1)^n/n=\lim_{x\to 1^-}\sum_n(-1)^nx^n/n,$$
d'où
$$\sum_n(-1)^n/n=\lim_{x\to1^-}-\ln(1+x)=-\ln(2),$$
la dernière égalité venant de la CONTINUITE DU LOGARITHME.
}

\comment{Vous pourrez aussi utiliser la série $\sum_n x^n/n$ mais en l'évaluant en $x=-1$.}

\comment{ATTENTION: dans l'énoncé du théorème d'Abel radial, on suppose la convergence de la série en UN SEUL point $z_0$ de module le rayon de convergence $\rho$, et on obtient la convergence uniforme sur le segment $[0,z_0]$. Ce résultat ne se généralise pas au cas où on est plus généreux et suppose la convergence de la série en TOUT point de module $\rho$ : dans ce cas-là la convergence est uniforme sur CHAQUE DIAMETRE FERME de $\overline{D}(0,1)$ (d'après Abel radial), mais la convergence pourrait être non uniforme sur $\overline{D}(0,1)$ --- vous trouverez des contre-exemples si vous tournez la page du sujet.}

\vspace{.1in}
\noindent
\textbf{4.}
Vrai ou faux ?
Soit $S(z)=\sum_n a_nz^n$ une série entière de rayon de convergence $1$.
\begin{enumerate}[label=\alph*)]
    % \item Si $S(1)$ converge, alors $\lim_{\substack{z\to 1\\z\in[0,1[}}S(z)$ existe.
    \item Si $\lim_{\substack{z\to 1\\z\in[0,1[}}S(z)$ existe, alors $S(1)$ converge.\\
    \correction{Faux. Contre-exemples: $\sum_n(-z)^n$, etc.}\\
    \comment{Ne confondez pas les deux convergences: la convergence de $S(x)$ lorsque $x\to 1^-$ et la convergence de la série numérique $S(1)$.}
    \item La série dérivée $S'(z)=\sum_n na_nz^{n-1}$ converge partout sur $D(0,1)$.\\
    \correction{Vrai. Parce que $S'(z)$ et $S(z)$ ont le même RAYON de convergence d'après le cours.}
    \item La série dérivée $S'(z)=\sum_n na_nz^{n-1}$ converge partout où $S(z)$ converge.\\
    \correction{Faux. A priori elles n'ont pas le même DOMAINE de convergence. Contre-exemple : $S(z)=\sum_nz^n/n$ et $z=-1$.}
    \item Si $S(z)$ converge uniformément sur $D(0,1)$, alors $S(z)$ converge sur $\overline{D}(0,1)$.\\
    \correction{Vrai. Il faudrait reprendre la même preuve que TD 2 Exercice 6 comme suit. Comme $S(z)$ converge uniformément sur $D(0,1)$, pour tout $\varepsilon>0$ il existe $N\in\N$ tel que pour tous $n>m\geq N$ et $z\in D(0,1)$, on a
    $$\abs{\sum_{k=m}^na_kz^k}\leq\varepsilon.$$
    On vérifie maintenant que cette inégalité rest valable pour $\abs{z}\leq 1$: si $\abs{z}<1$, il ne rest rien à faire ; si $\abs{z}=1$, prenons une suite $(t_l)_{l\in\N}\in[0,1[$ qui tend vers $1$ (par exemple $t_l=1-1/l$), alors $t_lz\in D(0,1)$, donc on a
    $$\abs{\sum_{k=m}^na_k\left(t_lz\right)^k}\leq\varepsilon;$$
    comme il n'y a qu'un nombre fini de termes et $l$ est indépendant de $m$ et $n$ et $k$, on peut faire tendre $l\to+\infty$ et trouver
    $$\abs{\sum_{k=m}^na_kz^k}\leq\varepsilon.$$
    Ainsi $S(z)$ vérifie le critère de Cauchy uniform sur $\overline{D}(0,1)$, d'où la convergence (uniforme) sur $\overline{D}(0,1)$.
    }\\
    \comment{Cela vous dit que on n'espère presque jamais de convergence UNIFORME sur le disque ouvert du rayon égal au rayon de convergence, puisque SOUVENT une série entière ne converge PAS PARTOUT SUR SON CERCLE DE CONVERGENCE.}
\end{enumerate}

\vspace{.1in}
\noindent
\textbf{5.}
Soit $f(z)=\sum_{n\geq0}a_nz^n$ une fonction DSE sur $\C$ tout entier. On se propose de montrer que 
\begin{quote}
    \textit{Si $f(z)$ est bornée sur $\C$ par une constante $M>0$, alors $f(z)$ est une constante.}
\end{quote}
\begin{enumerate}[label=\alph*)]
    % \item Rappeler pourquoi $f(z)$ est continue sur $\C$.
    \item Soient $R>0$ et $N\in\N$. Montrer que la série
    $$\sum_{n\geq0}a_nR^{n-N}e^{i(n-N)\theta},\quad \theta\in[0,2\pi]$$
    en la variable $\theta$ converge uniformément vers la fonction $\frac{f(Re^{i\theta})}{R^Ne^{iN\theta}}$.\\
    \correction{Le point est de montrer l'UNIFORMITE de la convergence.\\
    \textbf{Argument 1}: il suffit de montrer la convergence normale, c'est-à-dire la convergence de la série numérique $\sum_n \abs{a_n}R^{n-N}=R^{-N}\sum_n\abs{a_n}R^n$. Or, d'après le cours $\sum_n\abs{a_n}R^n$ converge si $R$ est STRICTEMENT inférieur au rayon de convergence, qui vaut $+\infty$ d'après l'hypothèse.}\\
    \comment{\textbf{Argument 2} (qui est essentiellement le même que le 1) : il suffit de montrer la convergence uniforme de la série $\sum_na_nR^ne^{in\theta}$ puis diviser par $R^Ne^{iN\theta}$. Cette série-ci est l'évaluation de la série $f(z)$ en $z=Re^{i\theta}$, i.e. sur le cercle de rayon $R$. Or, d'après le cours, comme $R$ est STRICTEMENT inférieur au rayon de convergence, qui vaut $+\infty$, on a la convergence UNIFORME sur le disque fermé $\overline{D}(0,R)$, en particulier sur le cercle de rayon $R$.}
    \item Calculer $\int_0^{2\pi}e^{in\theta}d\theta$ pour $n\in\Z$. En déduire que 
    $$\frac{1}{2\pi}\int_0^{2\pi}\frac{f(Re^{i\theta})}{R^Ne^{iN\theta}}d\theta=a_N.$$
    \correction{$\int_0^{2\pi}e^{in\theta}d\theta=(e^{2n\pi i}-1)/in=0$ si $n\neq 0$, et $\int_0^{2\pi}e^{in\theta}d\theta=2\pi$ si $n=0$.
    Puis, d'après la convergence uniforme en $\theta$ de la question précédente, on peut échanger l'intégration par rapport à $\theta$ et la somme $\sum_n$, puis utiliser le calcul qu'on vient de faire :
    $$\frac{1}{2\pi}\int_0^{2\pi}\frac{f(Re^{i\theta})}{R^Ne^{iN\theta}}d\theta=\frac{1}{2\pi}\sum_n\int_0^{2\pi}a_nR^{(n-N)}e^{i(n-N)\theta}d\theta=\frac{1}{2\pi}(a_N\cdot2\pi+\sum_{n\neq N}0)=a_N,$$
    où on voit l'importance du calcul de $\int_0^{2\pi}e^{in\theta}d\theta$.}
    \item Montrer que $\abs{a_N}\leq\frac{M}{R^N}$. En conclure.\\
    \correction{Par inégalité triangulaire, puis par la majoration $\norm{f}_\infty\leq M$, on obtient
    $$\abs{a_N}=\abs{\frac{1}{2\pi}\int_0^{2\pi}\frac{f(Re^{i\theta})}{R^Ne^{iN\theta}}d\theta}\leq
    \frac{1}{2\pi}\int_0^{2\pi}\abs{\frac{f(Re^{i\theta})}{R^Ne^{iN\theta}}}d\theta\leq
    \frac{1}{2\pi}\int_0^{2\pi}\frac{M}{R^N}d\theta=\frac{M}{R^N}.$$
    Ici $R>0$ étant ARBITRAIRE, donc en faisant tendre $R\to+\infty$, on obtient $\abs{a_N}\leq 0$ si $N>0$, donc $a_N=0$ si $N>0$, d'où $f(z)=\sum_n a_nz^n=a_0$ est une constante.}
\end{enumerate}  






\end{document}

