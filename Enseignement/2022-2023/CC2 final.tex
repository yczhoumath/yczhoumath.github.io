\documentclass[a4paper,10pt]{article}
\usepackage[french]{babel}
\usepackage[T1]{fontenc}
\usepackage[utf8]{inputenc}
\usepackage[margin=1in]{geometry}
\linespread{1.2}
\pagestyle{empty}
\usepackage{enumitem}

\usepackage{amsmath,amssymb,amsfonts}
\newcommand{\N}{\mathbb{N}}
\newcommand{\Z}{\mathbb{Z}}
\newcommand{\R}{\mathbb{R}}
\newcommand{\C}{\mathbb{C}}
\newcommand{\X}{\mathcal{X}}
\newcommand{\scalarprod}[2]{\langle#1,#2\rangle}
\newcommand{\norm}[1]{\|#1\|}
\newcommand{\abs}[1]{\left|#1\right|}
\let\Re\relax
\DeclareMathOperator{\Re}{Re} % real part

\begin{document}

\noindent LU2MA260 - Contrôle continu n\textdegree2
\hfill 13 décembre 2022
\medskip\hrule
\vspace{.3in}

\emph{Durée: 1 heure. Aucun document autorisé.
Ce contrôle sera noté sur \textbf{10 points}. Justifier TOUT.}


\vspace{.1in}
\noindent
\textbf{1.}
Déterminer le rayon de convergence des séries entières suivantes:
\begin{enumerate}[label=\alph*)]
\begin{minipage}{0.4\linewidth}
    \item $\sum_n 2^nz^n$,
    \item $\sum_n 2^nz^{2n}$,
    \item $\sum_n 2^nz^{n^2}$,
\end{minipage}
\begin{minipage}{0.4\linewidth}
    \item $\sum_n n^{\ln n}z^n$
    \item $\sum_n \frac{(2n)!}{(n!)^2}z^n$,
    \item $\sum_n n\sin(\frac{n\pi}{3})z^n$.
\end{minipage}
\end{enumerate}


\vspace{.1in}
\noindent
\textbf{2.} Exemples.
\begin{enumerate}[label=\alph*)]
    \item Donner une série entière dont le domaine de convergence est exactement $D(0,1)$.
    \item Donner une série entière dont le domaine de convergence est exactement $\overline{D}(0,1)$.
\end{enumerate}

\vspace{.1in}
\noindent
\textbf{3.}
En appliquant le théorème d'Abel radial,
montrer que $\sum_{n=1}^{+\infty} \frac{(-1)^n}{n}=-\ln(2)$.

\vspace{.1in}
\noindent
\textbf{4.}
Vrai ou faux ?
Soit $S(z)=\sum_n a_nz^n$ une série entière de rayon de convergence $1$.
\begin{enumerate}[label=\alph*)]
    % \item Si $S(1)$ converge, alors $\lim_{\substack{z\to 1\\z\in[0,1[}}S(z)$ existe.
    \item Si $\lim_{\substack{z\to 1\\z\in[0,1[}}S(z)$ existe, alors $S(1)$ converge.
    \item La série dérivée $S'(z)=\sum_n na_nz^{n-1}$ converge partout sur $D(0,1)$.
    \item La série dérivée $S'(z)=\sum_n na_nz^{n-1}$ converge partout où $S(z)$ converge.
    \item Si $S(z)$ converge uniformément sur $D(0,1)$, alors $S(z)$ converge sur $\overline{D}(0,1)$.
\end{enumerate}

\vspace{.1in}
\noindent
\textbf{5.}
Soit $f(z)=\sum_{n\geq0}a_nz^n$ une fonction DSE sur $\C$ tout entier. On se propose de montrer que 
\begin{quote}
    \textit{Si $f(z)$ est bornée sur $\C$ par une constante $M>0$, alors $f(z)$ est une constante.}
\end{quote}
\begin{enumerate}[label=\alph*)]
    % \item Rappeler pourquoi $f(z)$ est continue sur $\C$.
    \item Soient $R>0$ et $N\in\N$. Montrer que la série
    $$\sum_{n\geq0}a_nR^{n-N}e^{i(n-N)\theta},\quad \theta\in[0,2\pi]$$
    en la variable $\theta$ converge uniformément vers la fonction $\frac{f(Re^{i\theta})}{R^Ne^{iN\theta}}$.
    \item Calculer $\int_0^{2\pi}e^{in\theta}d\theta$ pour $n\in\Z$. En déduire que 
    $$\frac{1}{2\pi}\int_0^{2\pi}\frac{f(Re^{i\theta})}{R^Ne^{iN\theta}}d\theta=a_N.$$
    \item Montrer que $\abs{a_N}\leq\frac{M}{R^N}$. En conclure. 
\end{enumerate}

\vfill
\begin{center}
    NE TOURNEZ PAS LA PAGE SVP…
\end{center}

% \vspace{.1in}
\newpage
\textit{On pourrait aussi considérer la réciproque d'Exercice 4 (d) : si $S(z)$ converge sur $\overline{D}(0,1)$, alors $S(z)$ converge uniformément sur $D(0,1)$. Celle-ci est fausse, cependant le contre-exemple n'est pas facile à construire; voici une construction sous forme d'<< exercice >>.}

\vspace{.1in}
Considérons la série entière
$$S(z)=\sum_{k\in\N}a_kz^k,\quad \text{avec}\quad a_k=\sum_{n\geq 0}\delta_n\frac{1}{(1+i\varepsilon_n)^n}.$$
où $(\delta_n)_{n\in\N}$ et $(\varepsilon_n)_{n\in\N}$ sont deux suites dans $\R_{>0}$ tendant vers 0 telles que
\begin{quote}
    (ACV) La série numérique $\sum_{n}\frac{\delta_n}{\varepsilon_n}$ converge;
    
    ~(DV)~ La suite $(\frac{\delta_n}{\varepsilon_n^2})_{n}$ diverge vers $+\infty$.
\end{quote}
\textit{Cet exercice a pour but de montrer que $S(z)$ converge sur $\overline{D}(0,1)$ mais la convergence n'y est pas uniforme}.

\begin{enumerate}[label=\alph*)]
    \item Justifier que les $a_k$ sont bien définis. Donner un exemple de $(\delta_n)_n$ et $(\varepsilon_n)_n$ vérifiant \emph{toutes} les conditions ci-dessus, de sorte que cet exercice n'est pas vide ! (Indication: considérer des suites de type $1/n^\alpha$.)
    \item (Question sur des séries numériques.) Soit $\sum_{n\in\N} u_n$ une série numérique absolument convergente. Soit $(\lambda_n)_n$ une suite dans $D(0,1)$. Montrer que la suite numérique $(\sum_n u_n\lambda_n^N)_{N\in\N}$ est bien définie et que
    $$\lim_{N\to+\infty}\sum_{n}u_n\lambda_n^N=0.$$
    Vous pourrez montrer d'abord que $\abs{\sum_n u_n\lambda_n^N}\leq\sum_{n=0}^m\abs{a_n}\abs{\lambda_n}^N+\sum_{n=m+1}^{+\infty}\abs{a_n}$ pour tout $m\in\N$.
    \item Soit $z\in\overline{D}(0,1)$. Montrer que $\abs{1+i\varepsilon_n-z}\geq \varepsilon_n$ si $z=1$, et $\abs{1+i\varepsilon_n-z}\geq\Re(1-z)>0$ si $z\neq 1$.
    En déduire l'absolue convergence de $\sum_n\frac{\delta_n}{1+i\varepsilon_n-z}$ pour tout $z\in\overline{D}(0,1)$.
    \item Montrer que 
    $$\sum_{k=0}^Na_kz^k-\sum_{n\geq0}\frac{\delta_n}{1+i\varepsilon_n-z}=-\sum_{n\geq0}\frac{\delta_n}{1+i\varepsilon_n-z}\left(\frac{z}{1+i\varepsilon_n}\right)^{N+1}.$$
    En déduire que $S(z)$ converge partout sur $\overline{D}(0,1)$ et que sa somme fait
    $$S(z)=\sum_{n\geq 0}\frac{\delta_n}{1+i\varepsilon_n-z}.$$
    \item Soit $z\in\overline{D}(0,1)$. Montrer que $\Re\left(\frac{1}{1+i\varepsilon_n-z}\right)=\frac{\Re(1-z)}{\abs{1+i\varepsilon_n-z}^2}\geq0$ pour tout $n\in\N$.
    En déduire que
    $$\Re S(z)\geq \frac{\Re(1-z)}{\abs{1+i\varepsilon_n-z}^2},\quad \forall n\in\N.$$
    \item Posons $z_n=\frac{1+i\varepsilon_n}{\abs{1+i\varepsilon_n}}$ (qui est l'intersection de $[0,1+i\varepsilon_n]$ avec le cercle unité). Montrer que
    $$\frac{\Re(1-z_n)}{\abs{1+i\varepsilon_n-z_n}^2}\sim\frac{2\delta_n}{\varepsilon_n^2},\quad n\to+\infty.$$
    \item Montrer que $S(z)$ n'est pas bornée sur $\overline{D}(0,1)$. En conclure.
\end{enumerate}
    
    






\end{document}


En calculant les sommes partielles de $S(z)$ et la convergence dominée, on obtient que $S(z)$ converge sur $\overline{D}(0,1)$ et la somme fait
    $$S(z)=\sum_{n=0}^{+\infty}\frac{\delta_n}{1+i\varepsilon_n-z}.$$
    Ceci converge bien sûr aussi sur $\overline{D}(0,1)$.
    
     (on a rendu $1+i\varepsilon$ à avoir module $1$ en le dilatant). On calcule pour $z\in\overline{D}(0,1)$
    
    qui est une série numérique de termes positifs, donc en particulier
    $$\Re S(z_n)\geq\frac{\delta_n\Re(1-z_n)}{\abs{1+i\epsilon_n-z_n}^2}=\frac{\delta_n}{\abs{1+i\epsilon_n}(\abs{1+i\epsilon_n}-1)}\sim\frac{2\delta_n}{\epsilon_n^2}.$$
    On suppose maintenant qu'en plus de (ACV), on a
    \begin{quote}
        
    \end{quote}
    Alors $S(z)$ n'est pas bornée sur $\overline{D}(0,1)$ autour de $z=1$.
    Ainsi, la convergence de la série sur $\overline{D}(0,1)$ ne peut pas être uniforme; le rayon de convergence de $S(z)$ vaut bien 1.



