\documentclass[a4paper,10pt]{article}
\usepackage[french]{babel}
\usepackage[T1]{fontenc}
\usepackage[utf8]{inputenc}
\usepackage[margin=1in]{geometry}
\linespread{1.2}
\pagestyle{empty}
\usepackage{enumitem}

\usepackage{amsmath,amssymb,amsfonts}
\newcommand{\N}{\mathbb{N}}
\newcommand{\Z}{\mathbb{Z}}
\newcommand{\R}{\mathbb{R}}
\newcommand{\C}{\mathbb{C}}
\newcommand{\X}{\mathcal{X}}
\newcommand{\scalarprod}[2]{\langle#1,#2\rangle}
\newcommand{\norm}[1]{\|#1\|}
\newcommand{\abs}[1]{\left|#1\right|}

\usepackage{xcolor}
\newcommand{\correction}[1]{{\color{red}#1}}
\newcommand{\comment}[1]{{\color{blue}#1}}

\begin{document}

\noindent LU2MA260 - Contrôle continu n\textdegree1
\hfill 23 octobre 2022
\medskip\hrule
\vspace{.3in}

\emph{Durée: 1 heure. Aucun document autorisé.
Ce contrôle sera noté sur \textbf{10 points}.}

\correction{Corrigé en rouge (peut-être indication toute seule).}

\comment{Commentaire en bleu.}

\vspace{.1in}

\noindent
\textbf{1. (2 pts)} Questions de cours:

\begin{enumerate}[label=\alph*)]
    \item Rappelez la comparaison séries-intégrales, puis montrez la divergence de la série de Bertrand $\sum_{n\geq 2}\frac{1}{n\ln(n)}$.\\
    \correction{Soit $f(x)$ une fonction continue \textbf{positive décroissante vers 0} sur $[0,+\infty[$. Alors la série $\sum_{n\geq0} f(n)$ et l'intégrale $\int_0^{+\infty} f(t)dt$ sont de la même nature. On l'applique à $f(x)=\frac{1}{x\ln(x)}$ définie sur $[2,+\infty[$, en calculant
    \[\int_2^{+\infty}\frac{1}{t\ln(t)}dt=\int_2^{+\infty}\frac{1}{\ln(t)}d\ln(t)=\int_{\ln(2)}^{+\infty}\frac{1}{t}dt\]
    qui diverge.}
    \item Rappelez pourquoi si $(f_n)_n$ est une suite de fonctions bornées sur un intervalle $I$ qui converge uniformément vers $f$, alors $f$ est bornée sur $I$.\\
    \correction{Soit $N\in\N$ tel que $\abs{f_N(x)-f(x)} \leq 1$ pour tout $x\in I$. On a alors
    \[\abs{f(x)}\leq\abs{f_N(x)}+1\leq\norm{f_N}_\infty+1\]
    pour tout $x\in I$.}
    % \item Rappelez la définition d'une norme $\norm{\cdot}$ sur un $\R$-espace vectoriel $E$.
\end{enumerate}

\vspace{.1in}

\noindent
\textbf{2. (3 pts)}
Déterminer la nature des suites de nombres suivantes :

\begin{enumerate}[label=\alph*)]
    \item $\sum_{n\geq1}\frac{1}{n+\ln(n)^2}$.
    \correction{Divergente par équivalence : $\frac{1}{n+\ln(n)^2} \sim \frac{1}{n}$.}
    \item $\sum_{n\geq2}\left(\frac{\ln(n)}{n}\right)^2$. (Indication : comparaison avec une série de Riemann.)\\
    \correction{Convergente par comparaison : $\left(\frac{\ln(n)}{n}\right)^2 \sim o(\frac{1}{n^{3/2}})$.}
    \item $\sum_{n\geq2} n e^{-n^2+n}$.
    \correction{Convergente par le critère de d'Alembert : $\frac{(n+1) e^{-(n+1)^2+(n+1)}}{n e^{-n^2+n}}=\frac{n+1}{n} e^{-2n} \to 0$.}
\end{enumerate}

\vspace{.1in}

\noindent
\textbf{3. (2 pt)}
Déterminez si la \textbf{suite} de fonctions $f_n(x)=nxe^{-nx}$ converge uniformément sur $I=[1/10,+\infty[$, respectivement sur $I=[0,+\infty[$. 

\correction{Oui pour $I=[1/10,+\infty[$, car si $n\geq 10$, $\norm{f_n}_\infty=\frac{n}{10}e^{-\frac{n}{10}}\to 0$. Non pour $I=[0,+\infty[$, car $f_n$ converge simplement vers 0 mais $f_n(\frac{1}{n})=e^{-1}\not\to0$.}

\vspace{.1in}

\noindent
\textbf{4. (3 pts)} Vrai ou faux : jusitifiez si l'énoncé est vrai, donnez-en un contre-exemple s'il est faux.

\noindent
Soit $\sum_{n\geq0} a_n$ une série de nombres.
\begin{enumerate}[label=\alph*)]
    \item Si elle converge, alors $\lim_n\sqrt[n]{\abs{a_n}} \leq 1$ dès que la limite existe.\\
    \correction{Vrai, comme contraposé du critère de Cauchy, ou par comparaison avec la série géométrique.}
    \item Si elle converge, alors $\sum_{n\geq0} a_n^2$ converge. (Indicaiton : séries altérnées.)\\
    \correction{Faux, contre-exemple : $a_n=\frac{(-1)^n}{\sqrt{n}}$.}
    \item Si elle diverge et si $a_n\geq0$, alors $a_n \not\to 0,~n\to+\infty$.\\
    \correction{Faux, contre-exemple : $a_n=\frac{1}{n}$.}
\end{enumerate}
Soit $(f_n)_n\in\mathcal{C}^0([0,1],\C)^\N$ une \textbf{suite} de fonctions.
\begin{enumerate}[resume*]
    \item Si $f_n$ converge simplement vers une fonction $f\in\mathcal{C}^0([0,1],\C)$ et si $\int_0^1f_n \to \int_0^1 f$, alors la convergence $f_n \to f$ est uniforme sur $[0,1]$. (Indication : on peut supposer $f=0$.)\\
    \correction{Faux, contre-exemple : la suite dans l'exercice 3.}\\
    \comment{Si on impose la décroissance de la suite, i.e. $f_n(x)\geq f_{n+1}(x)$, alors la convergence de la suite des intégrales est due au théorème de convergence dominée, et la convergence uniforme est due au théorème de Dini.}
    \item Si $0 \leq f_n \leq 1$ et $f_{n+1}\leq f_n$, alors $f_n$ converge simplement.\\
    \correction{Vrai, par la convergence monotone.}
    \item Si $0 \leq f_n \leq 1$ et $f_{n+1}\leq f_n$, alors $f_n$ converge uniformément.\\
    \correction{Faux, contre-exemple : $f_n(x)=e^{-nx}$.}
\end{enumerate}

% \vspace{.1in}

% Déterminer la nature des séries numérique suivantes:
% \begin{enumerate}[label=\alph*)]
% \begin{minipage}{0.4\linewidth}
% \end{minipage}
% \begin{minipage}{0.4\linewidth}
% \end{minipage}
% \end{enumerate}


\vspace{.1in}

\noindent
\textbf{5. (3 pts)}
Considérons la \textbf{série} de fonctions $\sum_{n\geq1}(-1)^{n-1}\frac{x^n}{\sqrt{n}}$.
\begin{enumerate}[label=\alph*)]
    \item Montrez que si $x\in]-1,1]$, la série converge.\\
    \correction{Critère d'Abel pour $\sum_{n\geq1}(-1)^{n-1}x^n\frac{1}{\sqrt{n}}$.}
    % \item Montrez que si $x\in]-\infty,-1]$, la série diverge.
    % \item Montrez que si $x\in]1,+\infty]$, la série diverge.
    \item Montrez que la série converge uniformément sur $[-a,a]$ si $0<a<1$.\\
    \correction{En fait, elle converge uniformément.}
\end{enumerate}
Les trois questions suivantes sont étroitement reliées :
\begin{enumerate}[resume*]
    \item Écrivez le critère de Cauchy pour la convergence uniforme de la série (i.e. pour la convergence uniforme de la suite des sommes partielles de la série) sur $]-1,0]$.\\
    \correction{$\forall \epsilon>0, \exists N, \forall n\geq m\geq N, \forall x\in ]-1,0]$, on a
    \[\tag{$\star$} \abs{f_n(x)-f_m(x)}\leq \epsilon.\]}
    \item Qu'obtenez vous lorsque $x\to(-1)^+$ ? (Ceci devrait correspondre au critère de Cauchy pour une série de nombres que vous préciserez.)\\
    \correction{L'inégalité reste vraie pour $x=-1$. C'est le critère de Cauchy pour la série harmonique $\sum_{n\geq1} \frac{1}{n}$.}
    \item En déduisez que la convergence de la série n'est pas uniforme sur $]-1,0]$.\\
    \correction{S'il y avait la convergence uniforme sur $]-1,0]$, alors la série harmonique convergerait, ce qui est contradictoire.}
\end{enumerate}
Petite remarque : la convergence sur $[0,1]$ est pourtant uniforme (Théorème d'Abel radial).

\vfill

\centering\emph{Fin du corrigé.}

\end{document}


